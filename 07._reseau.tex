\documentclass{beamer}
\usetheme[compress]{Singapore}
%\useoutertheme{miniframes}

% \documentclass{beamer}
%\usetheme{Warsaw}

% Pour les documents en francais...
	\usepackage[latin1]{inputenc}
	\usepackage[french]{babel}
	\usepackage[french]{varioref}
%\usepackage[T1]{fontenc} 

% Math?matiques
	\usepackage{amsmath}

% Caracteres speciaux suppl?mentaires
	\usepackage{latexsym,amsfonts}

% A documenter
	\usepackage{moreverb}

% Macros pour les paquets
	\usepackage{array}  			% N?cessaires pour les tableaux de la macro Excel.

% Outil suppl?mentaire pour les tableaux
	\usepackage{multirow}
	\usepackage{booktabs}
	\usepackage{xcolor} % alternating row colors in table, incompatible avec certains modules
	\usepackage{longtable}
	\usepackage{colortbl}

% Pour ins?rer des graphiques
	\usepackage{graphicx} 			% Graphique simples
	\usepackage{subfigure}			% Graphiques multiples

% Pour ins?rer des couleurs
	\usepackage{color}

% Rotation des objets et des pages
%	\usepackage{rotating}
%	\usepackage{lscape}

% Pour insrer du code source, LaTeX ou SAS par exemple.
	\usepackage{verbatim}
         \usepackage{moreverb}
	\usepackage{listings}
	\lstset{basicstyle=\ttfamily,
  		showstringspaces=false,
  		commentstyle=\color{red},
  		keywordstyle=\color{blue}
	}
	\usepackage{fancyvrb}

%	\lstset{language=SAS,numbers=left}		% Par dfaut le listing est en SAS

% Pour ins?rer des hyperliens
  \usepackage{hyperref}

% American Psychological Association (for bibliographic references).
	\usepackage{apacite}

% Pour l'utilisation des macros
	\usepackage{xspace}

% Pour l'utilisation de notes en fin de document.
%	\usepackage{endnotes}

% Array
%	\usepackage{multirow}
%	\usepackage{booktabs}

% Rotation
%	\usepackage{rotating}

% En t?tes et pieds de pages
%	\usepackage{fancyhdr}
%	\usepackage{lastpage}


% Page layout

% By LaTeX commands
%\setlength{\oddsidemargin}{0cm}
%\setlength{\textwidth}{16cm}
%\setlength{\textheight}{24cm}
%\setlength{\topmargin}{-1cm}
%\setlength{\marginparsep}{0.2cm}

% fancyheader parameters
%\pagestyle{fancy}

%\fancyfoot[L]{{\small Formation \LaTeX, DEPP}}
%\fancyfoot[c]{}
%\fancyfoot[R]{{\small \thepage/\pageref{LastPage}}}

%\fancyhead[L]{}
%\fancyhead[c]{}
%\fancyhead[R]{}

% Pour ins?rer des dessins de Linux
\newcommand{\LinuxA}{\includegraphics[height=0.5cm]{Graphiques/linux.png}}
\newcommand{\LinuxB}{\includegraphics[height=0.5cm]{Graphiques/linux.png}\xspace}

% Macro pour les petits dessins pour les diff?rents OS.
\newcommand{\Windows}{\emph{Windows}\xspace}
\newcommand{\Mac}{\emph{Mac OS X}\xspace}

\newcommand{\Linux}{\emph{Linux}\xspace}
\newcommand{\linux}{\emph{Linux}\xspace}

\newcommand{\GNULinux}{\emph{GNU/Linux}\xspace}
\newcommand{\gnulinux}{\emph{GNU/Linux}\xspace}

\newcommand{\Fedora}{\emph{Feodra}\xspace}
\newcommand{\Ubuntu}{\emph{Ubuntu}\xspace}


\newcommand{\MikTeX}{MiK\tex\xspace}
\newcommand{\latex}{\LaTeX\xspace}


\newcommand{\df}{\emph{data.frame}\xspace}
\newcommand{\liste}{\emph{list}\xspace}
\newcommand{\cad}{c'est-�-dire\xspace}

% Titre
\title{Introduction � GNU/Linux}
\author{Pascal Bessonneau}
\institute{Starinux}
\date{11/2017}

\subtitle{R�seau}


\newcommand{\hreff}[2]{\underline{\href{#1}{#2}\xspace}}

\begin{document}

\begin{frame}
	\maketitle
\end{frame}

\begin{frame}
	\tableofcontents
\end{frame}

% Begin document %%%%%%%%%%%%%%%%%%%%%%%%%%%%%%%%%%%%%%%%%%%%%%%%%%%%%%%%%%%%%%%%%%%%%%%%%%%%%%%%%%

% tail head less more cat

\section{R�seau}

\subsection{D�boguer un r�seau}

\begin{frame}[containsverbatim]
  \frametitle{Je n'ai pas de connexion du tout}
		
	Si vous faites un \emph{ifconfig} et que l'interface r�seau ethernet (enX ou ethX) n'apparait pas\dots
	
	Votre controleur r�seau n'est pas reconnue par \GNULinux. Dans ce cas il faut regarder sur Internet comment l'activer
	pour \GNULinux. Ca peut �tre compliqu�.
	
	G�n�ralement il faut t�l�charger le firmware ou charger le bon module du noyau.
	
\end{frame}

\begin{frame}[containsverbatim]
  \frametitle{Je n'ai pas de connexion du tout}
		
	Si vous faites un \emph{ifconfig} une adresse internet n'est pas assign�e\dots
	
	Dans ce cas, il est probable que ce soit un cable ou que la box ne joue pas son r�le de DHCP. Le r�seau est indisponible.
	
	Essayer de jouer avec les cables, rebooter votre box, \dots
		
\end{frame}

\begin{frame}[containsverbatim]
  \frametitle{Je n'ai pas de connexion du tout}
		
	Si vous faites un \emph{ifconfig} une adresse internet n'est assign�e\dots
	
	La box joue son r�le de DHCP. Le r�seau est disponible. Essayer de faire \emph{ping 208.118.235.174}. \emph{208.118.235.174} est 
	le serveur de la Free Software Foundation qui r�pond au ping sauf cataclysme.
	
	Si vous obtenez une r�ponse alors c'est surement le serveur DNS qui est mal configur� si vous avez 
	un serveur DNS priv� ou vous avez un peu trop jou� avec les r�glages de resolv.conf ou un probl�me
	de pare-feu mal configur� sur la box ou votre routeur.
	
	Si vous n'avez pas de r�ponse au \emph{ping} alors vous n'�tes pas connect� � internet. Dans ce cas
	c'est souvent un probl�me	de pare-feu mal configur� sur la box ou votre routeur.
	
	Autre possibilit� si tout parait au vert mais que vous n'avez pas acc�s au web, v�rifier les r�glages Proxy de votre navigateur.
	
\end{frame}


\end{document}