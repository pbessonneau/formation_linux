\section{La th�orie}

  \subsection{Les syst�mes \GNULinux}

	Le propos sera simplifi� et pourra choquer des sp�cialistes. Les processus et leur gestion est un sujet complexe.
	
	Un processus est par exemple quand vous lancez une application. Elle poss�de son petit bout de m�moire propre, des informations comme le r�pertoire o� vous l'avez lanc�, le droit d'avoir un certain temps d'utilisation du CPU, \dots
	
	Sous \GNULinux, un processus  (une application au sens large) est toujours le fils d'un autre processus. Ainsi � chaque boot, le premier processus va se multiplier � chaque que n�cessaire avec des infos et des droits adapt�s.





	Cette hi�rarchie peut �tre vu avec la commande \emph{pstree}.




  \subsection{Au quotidien}

	Les �lements importants � savoir au quotidien est principalement que chaque processus se voit affecter des droits (par exemple root, ou pascal ou gaston) et qu'il a un num�ro unique qui l'identifie.

	Identifier les droits avec lequel est lanc� le processus est important car cela conditionne ce qu'on pourra faire au sein du processus.
	
	C'est la raison pour laquelle on utilise \emph{sudo}. On laisse en utilisant cette commande le processus en tant que root.


\section{Les signaux}
  \subsection{Les syst�mes \GNULinux}

	Un processus peut recevoir des signaux venant du noyau. Les signaux sont, en grande partie, la forme de communication pour g�rer les processus.
	
	Nous verrons ici les ordres envoy�s aux processus par le noyau~: ceux qui nous int�ressent sont ceux qui vont g�rer l'arr�t et le red�marrage des processus.
	
	Ainsi les signaux � m�moriser sont~:
	\begin{description}
		\item[TERM (15)] pour dire � l'application de se fermer
		\item[KILL (9)] pour tuer l'application 
		\item[HUP  (1)] pour red�marrer l'application (par exemple un serveur pour qu'il relise ses fichiers de configurations)
	\end{description}
	

  \subsection{Au quotidien}

	Le num�ro de processus va permettre par exemple s'il se bloque de l'arr�ter ou de le tuer en utilisant une commande et le num�ro du processus.
	
	Oui sous \GNULinux on tue des processus, c'est cruel.
	

\section{En mode graphique}

  \subsection{Terminal/Console}
	
	\begin{description}
		\item[GNOME]\emph{Moniteur Syst�me}.
		\item[KDE] \emph{KDE System Guard} dans Syst�me.
	\end{description}
	

\section{En mode console}

  \subsection{Au quotidien}

	C'est la version simple. Si votre processus a un nom unique, la commande permet de tuer les processus. Par exemple pour stopper le navigateur Firefox vous pouvez taper~:
	\begin{verbatim} 
	pkill firefox
	\end{verbatim}
	
	S'il ne r�pond pas du tout vous pouvez le tuer~:
	\begin{verbatim} 
	pkill -9 firefox
	\end{verbatim}
	



\begin{verbatim}
pascal   31945  105  5.6 9244360 903364 ?      Sl   19:32   0:23 /usr/lib/firefox/firefox
pascal   32073  0.0  0.0  14376  1088 pts/0    S+   19:33   0:00 grep --color=auto firefox
\end{verbatim}	

on voit que Firefox est le processus 31945 et qu'il est en sommeil.
	


	Pour le stopper, \emph{kill 31945} ou le tuer \emph{kill -9 31945}.

	Vous pouvez visualiser les t�ches en arborescence avec ps � l'aide de cette commande~: \emph{ps aux -ejHu}
		


	Il y a aussi l'utilitaire \emph{top} il permet d'afficher en temps (presque) r�el les processus actifs.

	Vous pouvez classer diff�remment en utilisant les touches~:
	\begin{itemize}
		\item[F] trie selon une colonnes diff�rentes
		\item[u] n'affiche que cet utilisateur
		\item[k] stoppe un processus
		\item[q] quitter \emph{top}
	\end{itemize}


