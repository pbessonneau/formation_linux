\section{R�seau}

\subsection{D�boguer un r�seau}

  \subsection{Je n'ai pas de connexion du tout}
		
	Si vous faites un \emph{ifconfig} et que l'interface r�seau ethernet (enX ou ethX) n'apparait pas\dots
	
	Votre controleur r�seau n'est pas reconnue par \GNULinux. Dans ce cas il faut regarder sur Internet comment l'activer
	pour \GNULinux. Ca peut �tre compliqu�.
	
	G�n�ralement il faut t�l�charger le firmware ou charger le bon module du noyau.
	

		
	Si vous faites un \emph{ifconfig} une adresse internet n'est pas assign�e\dots
	
	Dans ce cas, il est probable que ce soit un cable ou que la box ne joue pas son r�le de DHCP. Le r�seau est indisponible.
	
	Essayer de jouer avec les cables, rebooter votre box, \dots
		

		
	Si vous faites un \emph{ifconfig} une adresse internet n'est assign�e\dots
	
	La box joue son r�le de DHCP. Le r�seau est disponible. Essayer de faire \emph{ping 4.4.4.4}. \emph{4.4.4.4} est un
	serveur de Google qui r�pond au ping sauf cataclysme.
	
	Si vous obtenez une r�ponse alors c'est surement le serveur DNS qui est mal configur� si vous avez 
	un serveur DNS priv� ou vous avez un peu trop jou� avec les r�glages de resolv.conf ou un probl�me
	de pare-feu mal configur� sur la box ou votre routeur.
	
	Si vous n'avez pas de r�ponse au \emph{ping} alors vous n'�tes pas connect� � internet. Dans ce cas
	c'est souvent un probl�me	de pare-feu mal configur� sur la box ou votre routeur.
	
	Autre possibilit� si tout parait au vert mais que vous n'avez pas acc�s au web, v�rifier les r�glages Proxy de votre navigateur.
	


